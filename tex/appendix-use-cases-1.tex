\documentclass[document.tex]{subfiles}
\begin{document}

\chapter {Detailed Use Cases for the Fourth~Year~Project~System}
\label {ch:detailed-use-cases-1}

\sectionnote{AC}
\section {Setup}
\label {sec:detailed-use-cases-1-setup}

\begin{table}[!htbp]
  \centering
  \caption{Use case description for the ``Setup New Year'' use case of the fourth-year project management system.}
  \label{tbl:use-case-start-new-year}

  \begin{usecase}[Setup New Year]
    \ucpart{Description}
    Coordinators can set up a new year. This deletes all projects and Group Member accounts. It guides the Coordinator through a series of steps to set up the new year which include ensuring the correct supervisor accounts are in place, and setting deadlines
    %
    \ucpart{Actors}
    Project Coordinator
    %
    \ucpart{Preconditions}
    Setting up the first year OR previous year is finished (all deadlines are in the past)
    %
    \ucpart{Flow}
    \ucnormal
    \begin{ucenum}
      \item The Coordinator selects to start a new year and a confirmation message is displayed (It must mention that all projects and group member accounts will be deleted)
      \item The Coordinator agrees and all projects and group member accounts are deleted
      \item The Coordinator is presented with a list of all current supervisor accounts and is able to add and delete them before moving onto the next step
      \item The Coordinator chooses to move to the next step and is presented with a list of all deadlines and must schedule them all before moving on to the next step
      \item The Coordinator schedules all the deadlines and the New Year has started
    \end{ucenum}
    %
    \ucpart{Variations}
    \ucbranch{A}
    \begin{ucenum*}
      \item [A.2] The Coordinator disagrees with starting a new year and is sent back to the default application page
    \end{ucenum*}
    %
    \ucpart{Exceptions}
    \ucexception{Deadline Not Set}
    All deadlines must be set, the Coordinator is sent back to step 4
    %
    \ucpart{Postconditions}
    The new year is ready and all deadlines are scheduled
  \end{usecase}
\end{table}


\begin{table}[!htbp]
  \centering
  \caption{Use case description for the ``Schedule Deadline'' use case of the fourth-year project management system.}
  \label{tbl:use-case-reschedule-deadline}

  \begin{usecase}[Schedule Deadline]
    \ucpart{Description}
    Deadlines can be rescheduled as long as they have not already been passed
    %
    \ucpart{Actors}
    Project Coordinator
    %
    \ucpart{Preconditions}
    The deadline that is being changed is in the future and has not already passed
    %
    \ucpart{Flow}
    \ucnormal
    \begin{ucenum}
      \item A Coordinator selects to schedule deadlines and is presented with a list of all possible deadlines and their dates
      \item A Coordinator selects a deadline to reschedule and is presented with a way to input a new date
      \item The Coordinator selects a new date and agrees to the scheduling of the deadline via a confirmation box
      \item The deadline has been scheduled
    \end{ucenum}
    %
    \ucpart{Exceptions}
    \ucexception{Date selected is in the past}
    You cannot assign a deadline in the past, the coordinator is sent back to step 3
    %
    \ucpart{Postconditions}
    The deadline is scheduled
  \end{usecase}
\end{table}


\FloatBarrier



\sectionnote{AC}
\section{Account Management}
\label{sec:account-management}

The account management use cases shown in Figure \ref{fig:case-4ys-use-case-account-management} are described in the following tables \ref{tbl:use-case-register-member-account} through \ref{tbl:use-case-manage-account}. These descriptions cover the creation, modification, and deletion of the different accounts in the system.

\begin{table}[!htbp]
  \centering
  \caption{Use case description for the ``Register Group Member Account'' use case of the fourth-year project management system.}
  \label{tbl:use-case-register-member-account}

  \begin{usecase}[Register Group Member Account]
    \ucpart{Description}
    A student is able to register their own account
    %
    \ucpart{Actors}
    Project Group Member
    %
    \ucpart{Description}
    \ucnormal
    \begin{ucenum}
      \item The user selects to register a new account
      \item The user provides an email address, password, confirmation password, full name, student number, and programme
      \item The system validates and saves the account
    \end{ucenum}
    %
    \ucpart{Exceptions}
    \ucexception{Email not unique}
    An account with that email address already exists and the user is redirected to step 2 and notified of the error
    \ucexception{Password and confirmation password do not match}
    User is redirected to step 2 and notified of the error
    \ucexception{Name, email, student number, programme, password or confirmation password is not provided}
    User is redirected to step 2 and notified of the error
    \ucexception{Student number not unique}
    An account with that student number already exists and the user is redirected to step 2 and notified of the error
    %
    \ucpart{Postconditions}
    If the Group Member Account was created successfully they will be logged in. If the Group Member Account was not created successfully they will be prompted to attempt to create their account again.
  \end{usecase}
\end{table}


\begin{table}[!htbp]
  \centering
  \caption{Use case description for the ``Create Supervisor Account'' use case of the fourth-year project management system.}
  \label{tbl:use-case-create-supervisor}

  \begin{usecase}[Create Supervisor Account]
    \ucpart{Description}
    Coordinators are able to create a Supervisor account by specifying the name, email, and password that the Supervisor will use to log in
    %
    \ucpart{Actors}
    Project Supervisor, Project Coordinator
    %
    \ucpart{Flow}
    \ucnormal
    \begin{ucenum}
      \item A Coordinator selects to create a new Supervisor Account
      \item The Coordinator provides a name, email, and password
      \item The system validates and saves the supervisor account
    \end{ucenum}
    %
    \ucpart{Exceptions}
    \ucexception{Email not unique}
    An account with that email address already exists and the user is redirected to step 2 and notified of the error
    \ucexception{Password and confirmation password do not match}
    User is redirected to step 2 and notified of the error
    \ucexception{Name, email, password or confirmation password is not provided}
    User is redirected to step 2 and notified of the error
    %
    \ucpart{Postconditions}
    The Supervisor account is created and can now be used
  \end{usecase}
\end{table}


\begin{table}[!htbp]
  \centering
  \caption{Use case description for the ``Delete Supervisor Account'' use case of the fourth-year project management system.}
  \label{tbl:use-case-delete-supervisor}

  \begin{usecase}[Delete Supervisor]
    \ucpart{Description}
    Coordinators are able to delete Supervisor accounts
    %
    \ucpart{Actors}
    Project Supervisor, Project Coordinator
    %
    \ucpart{Flow}
    \ucnormal
    \begin{ucenum}
      \item A Coordinator clicks on 'Delete' button next to the Supervisor they wish to delete
      \item A confirmation message is displayed requesting confirmation before deleting the supervisor
      \item The Coordinator agrees and the supervisor account is deleted
    \end{ucenum}
    %
    \ucpart{Variations}
    \ucbranch{A}
    \begin{ucenum*}
      \item [A.1] The Coordinator disagrees, the supervisor account is not deleted, and the Coordinator is sent back to the list of Supervisors
    \end{ucenum*}
    %
    \ucpart{Exceptions}
    \ucexception{Supervisor does not exist}
    The Supervisor that is attempting to be deleted does not exist, The coordinator is sent back to the list of Supervisors
    \ucexception{Supervisor is last supervisor of a project}
    The Supervisor cannot be deleted if they are the last supervisor in a project
    %
    \ucpart{Postconditions}
    The Supervisor account is deleted and is removed from all projects they were supervising.
  \end{usecase}
\end{table}


\begin{table}[!htbp]
  \centering
  \caption{Use case description for the ``Edit Account'' use case of the fourth-year project management system.}
  \label{tbl:use-case-manage-account}

  \begin{usecase}[Edit Account]
    \ucpart{Description}
    Any user may edit their own account. Only Project Coordinators are able to edit accounts other than their own.
    %
    \ucpart{Actors}
    Project Group Member, Project Supervisor, Project Coordinator
    %
    \ucpart{Preconditions}
    Account being edited exists
    %
    \ucpart{Flow}
    \ucnormal
    \begin{ucenum}
      \item User provides new account information
      \item The system validates and saves the edited account
    \end{ucenum}
    %
    \ucpart{Variations}
    \ucbranch{A}
    \begin{ucenum}
      \item [A.1] User selects to delete the account
      \item [A.2] The system deletes the account and all references to it
    \end{ucenum}
    %
    \ucpart{Exceptions}
    \ucexception{No privilege to edit\slash delete selected account}
    The account cannot be edited\slash deleted by the current user
    %
    \ucpart{Postconditions}
    If the Account was edited the changes are saved by the system. If the Account was deleted all references to that account and the account itself are removed.
  \end{usecase}
\end{table}


\FloatBarrier


\sectionnote{AC}
\section {Project Selection}
\label {sec:detailed-use-cases-1-management}


\begin{table}[!htbp]
  \centering
  \caption{Use case description for the ``Post Project'' use case of the fourth-year project management system.}
  \label{tbl:use-case-create-project}

  \begin{usecase}[Post Project]
    \ucpart{Description}
    A Project Coordinator is able to create a project with specified options and assign one or more supervisors. A Project Supervisor is allowed to create a project to which they are automatically signed.
    %
    \ucpart{Actors}
    Project Supervisor, Project Coordinator
    %
    \ucpart{Flow}
    \ucnormal
    \begin{ucenum}
      \item A Project Coordinator or a Project Supervisors selects to create a new project
      \item If the creator is a Project Coordinator a name, description, programmes, and supervisor are provided
      \item The system validates and saves the project and the provided supervisor is assigned to the project
    \end{ucenum}
    %
    \ucpart{Variations}
    \ucbranch{A}
    \begin{ucenum}
      \item [A.2] If the creator is a Project Supervisor a name, description, and programmes are provided.
      \item [A.3] The system validates and saves the project and the creator is assigned to the project
    \end{ucenum}
    %
    \ucpart{Exceptions}
    \ucexception{Name not unique}
    A Project with that name already exists and the user is redirected to step 2 and notified of the errors
    \ucexception{Name, description, programmes, or supervisor is not provided}
    The user is redirected to step 2 and notified of the errors
    %
    \ucpart{Postconditions}
    The Project is created and saved by the system
  \end{usecase}
\end{table}


\begin{table}[!htbp]
  \centering
  \caption{Use case description for the ``Remove Project'' use case of the fourth-year project management system.}
  \label{tbl:use-case-delete-project}

  \begin{usecase}[Remove Project]
    \ucpart{Description}
    A Project Coordinator is allowed to delete any project. A Project Supervisor is allowed to delete any project that they are assigned to.
    %
    \ucpart{Actors}
    Project Supervisor, Project Coordinator
    %
    \ucpart{Preconditions}
    The project that the Project Coordinator wishes to delete exists
    %
    \ucpart{Flow}
    \ucnormal
    \begin{ucenum}
      \item A Project Coordinator or Project Supervisor selects to delete a selected project
      \item A confirmation message is displayed notifying the user of what this operation means
      \item The user agrees and the Project is deleted and the user is redirected to viewing a list of all available projects
    \end{ucenum}
    %
    \ucpart{Variations}
    \ucbranch{A}
    \begin{ucenum}
      \item [A.3] The user disagrees and the user is redirected back to viewing the selected project
    \end{ucenum}
    %
    \ucpart{Exceptions}
    \ucexception{Permission Denied}
    The user is a Project Supervisor and is not assigned to the selected project and is returned to viewing the project
    %
    \ucpart{Postconditions}
    The Project is deleted
  \end{usecase}
\end{table}


\begin{table}[!htbp]
  \centering
  \caption{Use case description for the ``Browse Projects'' use case of the fourth-year project management system.}
  \label{tbl:use-case-view-project}

  \begin{usecase}[Browse Projects]
    \ucpart{Description}
    Project Group Members examine project details to determine which project they would like to work on.
    %
    \ucpart{Actors}
    Project Group Member
    %
    \ucpart{Flow}
    \ucnormal
    \begin{ucenum}
      \item The Project Group Member views the list of projects.
      \item The Project Group Member selects a project to view in more detail.
      \item The Project Group Member is presented with the project's name and description, as well as the programmes, Project Supervisors, and Project Group Members associated with the project.
    \end{ucenum}
  \end{usecase}
\end{table}


\begin{table}[!htbp]
  \centering
  \caption{Use case description for the ``Add Project Participant'' use case of the fourth-year project management system.}
  \label{tbl:use-case-join-project}

  \begin{usecase}[Add Project Participant]
    \ucpart{Description}
    A Project Supervisor can add any Project Group Member, which is not currently a member of another project, to the project that they are assigned to. Project Supervisors are able to add other Project Supervisors to projects they are assigned to.
    %
    \ucpart{Actors}
    Project Group Member, Project Supervisor
    %
    \ucpart{Preconditions}
    The project that is being joined exists
    %
    \ucpart{Flow}
    \ucnormal
    \begin{ucenum}
      \item The Project Supervisor selects to add a group member to the project
      \item A list of all group members not assigned to any project is displayed and one is selected
      \item The selected user is added to the project
    \end{ucenum}
    %
    \ucpart{Variations}
    \ucbranch{A}
    \begin{ucenum}
      \item [A.1] The Project Supervisor selects to add another supervisor to the project
      \item [A.2] A list of all Project Supervisors is displayed and one is selected, continue to 3
    \end{ucenum}
    %
    \ucpart{Exceptions}
    \ucexception{Project Group Member already in project}
    The selected Project Group Member is already assigned to a project (including this one). The user is returned to step 2
    \ucexception{Project Supervisor already in project}
    The selected Project Supervisor is already assigned to the current project. The user is returned to step 2A
    %
    \ucpart{Postconditions}
    The user is added to the project
  \end{usecase}
\end{table}


\begin{table}[!htbp]
  \centering
  \caption{Use case description for the ``Remove Project Participant'' use case of the fourth-year project management system.}
  \label{tbl:use-case-leave-project}

  \begin{usecase}[Remove Project Participant]
    \ucpart{Description}
    The Project Supervisor can remove any Project Group Member currently assigned to their project. The Project Supervisor can leave the project themselves if there is currently another Project Supervisor also assigned.
    %
    \ucpart{Actors}
    Project Group Member, Project Supervisor
    %
    \ucpart{Preconditions}
    The user being removed from the project is currently a member of the project
    %
    \ucpart{Flow}
    \ucnormal
    \begin{ucenum}
      \item A Project Supervisor selects a Project Group Member to be removed from the group
      \item A confirmation message is displayed notifying the user that they are removing the selected user
      \item The Project Supervisor agrees and the selected user is removed from the group
    \end{ucenum}
    %
    \ucpart{Variations}
    \ucbranch{A}
    \begin{ucenum*}
      \item [A.1] A Project Supervisor selects to leave the group themselves, continue from step 2
    \end{ucenum*}
    \ucbranch{B}
    \begin{ucenum}
      \item [B.3] The Project Supervisor disagrees and is returned to the view project screen
    \end{ucenum}
    %
    \ucpart{Exceptions}
    \ucexception{User Not In Project}
    The selected user is not in the project. The user is returned to the default view of the project.
    \ucexception{Last Supervisor}
    The last Project Supervisor in a project is not allowed to leave. The user is returned to the default view of the project.
    %
    \ucpart{Postconditions}
    The selected user is no longer a member of the project
  \end{usecase}
\end{table}


\FloatBarrier
\sectionnote{BM}
\section {Project Execution}
\label {sec:detailed-use-cases-1-execution}

\begin{table}[!htbp]
  \centering
  \caption{Use case description for the ``Produce Proposal'' use case of the fourth-year project management system.}
  \label{tbl:use-case-produce-proposal}

  \begin{usecase}[Produce Proposal]
    \ucpart{Description}
    Project Group Members produce a project proposal with feedback from their Project Supervisor
    %
    \ucpart{Actors}
    Project Group Member, Project Supervisor
    %
    \ucpart{Preconditions}
    The Project Supervisor has accepted the Project Group Members, but a project proposal has not been completed.
    %
    \ucpart{Flow}
    \ucnormal
    \begin{ucenum}
      \item A Project Group Member uploads and submits a PDF document of the proposal (which for now is a draft.)
      \item The system notifies the Project Supervisor and the other Project Group Members.
      \item The Project Supervisor reviews the proposal draft.
      \item The Project Supervisor provides feedback to the Project Group Members on the proposal draft.
      \item Return to step 1.
    \end{ucenum}
    %
    \ucpart{Variations}
    \ucbranch{A}
    \begin{ucenum}
      \item [A.4] The Project Supervisor accepts the proposal draft.
      \item [A.5] The system marks the draft as the final proposal and notifies the Project Group Members.
    \end{ucenum}
    %
    \ucpart{Exceptions}
    \ucexception{Deadline expires}
    If the proposal is not submitted and finalized before the deadline, it is marked as overdue. The interpretation of this status is left to the Project Supervisor.
    %
    \ucpart{Postconditions}
    A PDF of the proposal is stored and marked as accepted, and project has entered the next phase.
  \end{usecase}
\end{table}


\begin{table}[!htbp]
  \centering
  \caption{Use case description for the ``Submit Progress Report'' use case of the fourth-year project management system.}
  \label{tbl:use-case-submit-progress-report}

  \begin{usecase}[Submit Progress Report]
    \ucpart{Description}
    Project Group Members submit a status update document to their Project Supervisor summarizing their progress on their project.
    %
    \ucpart{Actors}
    Project Group Member, Project Supervisor
    %
    \ucpart{Preconditions}
    The Project Supervisor has accepted a proposal from this group, but has yet to accept a progress report.
    %
    \ucpart{Flow}
    \ucnormal
    \begin{ucenum}
      \item A Project Group Member uploads and submits a PDF document of the progress report.
      \item The system notifies the Project Supervisor and the other Project Group Members.
      \item The Project Supervisor reviews the progress report.
      \item The Project Supervisor accepts the progress report.
    \end{ucenum}
    %
    \ucpart{Variations}
    \ucbranch{A}
    \begin{ucenum}
      \item [A.4] The Project Supervisor returns the progress report.
      \item [A.5] Return to step 1.
    \end{ucenum}
    %
    \ucpart{Exceptions}
    \ucexception{Deadline expires}
    If the progress report is not submitted and finalized before the deadline, it is marked as overdue. The interpretation of this status is left to the Project Supervisor.
    %
    \ucpart{Postconditions}
    A PDF of the progress report is stored, and the project enters the next phase.
  \end{usecase}
\end{table}


\begin{table}[!htbp]
  \centering
  \caption{Use case description for the ``Schedule Group Oral Presentation'' use case of the fourth-year project management system.}
  \label{tbl:use-case-schedule-group-oral}

  \begin{usecase}[Schedule Group Oral Presentation]
    \ucpart{Description}
    Project Group Members and their Project Supervisor indicate times that they will be available for the oral presentation.
    %
    \ucpart{Actors}
    Project Group Member, Project Supervisor
    %
    \ucpart{Preconditions}
    The Project Supervisor has accepted a proposal from this group, and the oral presentation scheduling deadline has not passed.
    %
    \ucpart{Flow}
    \ucnormal
    \begin{ucenum}
      \item A Project Group Member fills out the Oral Presentation form.
      \item The system notifies the Project Supervisor and the other Project Group Members.
      \item The Project Supervisor and other Project Group Members review the scheduling form.
      \item The Project Supervisor and all of Project Group Members accept the submitted scheduling form.
      \item The group’s scheduling form is marked is approved.
      \item The oral presentation scheduling deadline expires.
    \end{ucenum}
    %
    \ucpart{Variations}
    \ucbranch{A}
    \begin{ucenum*}
      \item [A.4] Instead of accepting the schedule, a Project Group Member or the Project Supervisor changes the scheduling form.
      \item [A.5] Return to step 2.
    \end{ucenum*}
    \ucbranch{B}
    \begin{ucenum}
      \item [B.6] A Project Group Member or the Project Supervisor changes the scheduling form.
      \item [B.7] Return to step 2.
    \end{ucenum}
    %
    \ucpart{Postconditions}
    The last approved version of the scheduling form is recorded as the scheduling form submitted by the group. If no scheduling form was approved, then a form is submitted with all time slots marked as available.
  \end{usecase}
\end{table}


\begin{table}[!htbp]
  \centering
  \caption{Use case description for the ``Schedule Oral Presentations'' use case of the fourth-year project management system.}
  \label{tbl:use-case-schedule-orals}

  \begin{usecase}[Schedule Oral Presentations]
    \ucpart{Description}
    The Project Coordinator determines when and where each group will have their oral presentation.
    %
    \ucpart{Actors}
    Project Coordinator
    %
    \ucpart{Preconditions}
    The oral presentation scheduling deadline has passed.
    %
    \ucpart{Flow}
    \ucnormal
    \begin{ucenum}
      \item The Project Coordinator views each groups scheduling form.
      \item The Project Coordinator produces a schedule that best meets the constraints of all of the project groups, assigning each group a time and place for their oral presentation.
      \item The system notifies all Project Group Members and Project Supervisors.
      \item Project Group Members and Project Supervisors view the times and locations of their assigned oral presentations.
    \end{ucenum}
    %
    \ucpart{Exceptions}
    \ucexception{Rescheduling required}
    If a Project Group Member or Project Supervisor is no longer available for the timeslot they were assigned, they must contact the Project Coordinator. The Project Coordinator may choose to repeat the process from step 1 with the new scheduling constraints.
    %
    \ucpart{Postconditions}
    Each group has been assigned an appropriate time and location for their oral presentation.
  \end{usecase}
\end{table}


\begin{table}[!htbp]
  \centering
  \caption{Use case description for the ``Submit Poster Fair Demo Form'' use case of the fourth-year project management system.}
  \label{tbl:use-case-poster-fair-form}

  \begin{usecase}[Submit Poster Fair Demo Form]
    \ucpart{Description}
    The Project Group Members and Project Supervisor may opt to request space for a demonstration at the poster fair.
    %
    \ucpart{Actors}
    Project Group Members, Project Supervisor
    %
    \ucpart{Preconditions}
    The group’s oral presentation deadline has passed, but the date of the poster fair has not.
    %
    \ucpart{Flow}
    \ucnormal
    \begin{ucenum}
      \item A Project Group Member fills out and submits the poster fair demo request form.
      \item The system notifies the Project Group Members, Project Supervisor, and Project Coordinator.
    \end{ucenum}
    %
    \ucpart{Postconditions}
    The poster fair demo request form has been sent to the Project Coordinator.
  \end{usecase}
\end{table}


\begin{table}[!htbp]
  \centering
  \caption{Use case description for the ``Produce Final Report'' use case of the fourth-year project management system.}
  \label{tbl:use-case-final-report}

  \begin{usecase}[Produce Final Report]
    \ucpart{Description}
    Project Group Members produce a final project report with feedback from their Project Supervisor
    %
    \ucpart{Actors}
    Project Group Member, Project Supervisor, Project Coordinator
    %
    \ucpart{Preconditions}
    The project group has completed their oral presentation, but the final report has not yet been completed. Additionally, the submission deadline for the final report has not passed.
    %
    \ucpart{Flow}
    \ucnormal
    \begin{ucenum}
      \item A Project Group Member uploads and submits a PDF document of the final report (which for now is a draft.)
      \item The system notifies the Project Supervisor and the other Project Group Members.
      \item The Project Supervisor reviews the draft report.
      \item The Project Supervisor provides feedback to the Project Group Members on the draft report.
      \item Return to step 1.
    \end{ucenum}
    %
    \ucpart{Variations}
    \ucbranch{A}
    \begin{ucenum}
      \item [A.4] The Project Supervisor accepts the draft report.
      \item [A.5] The system marks the draft as the final report and notifies the Project Group Members.
      \item [A.6] The system forwards a copy of the final report to the Project Coordinator for distribution to the second reader.
    \end{ucenum}
    %
    \ucpart{Exceptions}
    \ucexception{Deadline expires}
    Submission is closed, and no final report can be submitted after the deadline. If a submission was pending acceptance, it is marked as accepted and proceeds to step A.5.
    %
    \ucpart{Postconditions}
    The project is completed. If the final report was submitted, then the PDF of the final report is stored and marked as accepted, and the Project Coordinator has received a copy.
  \end{usecase}
\end{table}


\FloatBarrier

\end{document}
