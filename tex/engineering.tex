\documentclass[document.tex]{subfiles}
\begin{document}


\chapter{The Engineering Project}
\label {ch:engineering}

\sectionnote {AC}
\section{Health and Safety}

This project defines a workflow development platform for creating web applications using Ruby on Rails. Because of the nature of this project there are no health and safety concerns.

\sectionnote {AC}
\section{Engineering Professionalism}

The main goal of this project was to define a platform from which workflow based web applications could be created from. Communication is very important as we must be able to describe in detail the steps that a developer needs to take in order to create a workflow based web application. We provided sufficient documentation for all of the libraries we created and made note of all the existing libraries that we found useful. In addition we have two case studies which provide examples of how each of our libraries work together to easily create a workflow based web application.

Because we have used existing libraries created by other developers licenses were important. Both the third-party libraries and our newly-created libraries are available under licenses that allow web applications using our platform to be distributed and sold.

\sectionnote {AC}
\section{Project Management}

At the very beginning of the project it was decided that we would test existing workflow frameworks and then build two case studies in an attempt to define common elements and define a standard way to create basic workflow projects. For each case study we used an iterative incremental development process. Iterative incremental development allowed us to gather requirements and test the systems as we built them. This allowed us to identify features that were common between all workflow applications during development and extract the functionalities as we went through the project.

We used the version control system git \cite{git} and github \cite{github} to manage all the code for this project. We made separate repositories for each library we created allowing us to keep our case studies and our libraries separate. This setup allows each library to be individually  maintained separately from the case studies which use them. 


\section{Individual Contributions}

\sectionnote {AC}
\subsection{Project Contributions}

The first part of our project was to test existing workflow-based frameworks for web applications. Brendan researched the Stonepath framework and Alexander researched the Ruote framework. We each created and designed a simple prototype application using the frameworks we researched to see how useful they actually were.

For the case study application Brendan worked on the state machine design and implementation while Alexander worked on account management and project selection. Detailed use cases were created by each member for what they worked on. The testing of all workflow-related behavior was done by Brendan and the integration testing of the rest of the system was done by Alexander.

After completing the case study, a platform for workflow-based web development was constructed from existing libraries and reusable functionality extracted from the platform. Brendan researched and found all of the libraries we used to construct our case studies and also extracted the common functionality of the case studies into their own libraries. Alexander used these libraries using only the documentation in order to ensure they were easy to understand and learn for a new developer.

For the proof-of-concept application, the roles were reversed from those in the case study. Alexander designed and implemented the state machines while Brendan worked on the project management functionality and most of the user interface. Alexander did the state machine testing and integration testing for the second case study.

\sectionnote {AC}
\subsection{Report Contributions}

The author of each section is identified by the initials to the left of the section heading. For example, the initials ``AC'' beside the heading above indicate that it was written by Alexander Clelland.

\end{document}
