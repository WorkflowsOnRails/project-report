\documentclass[document.tex]{subfiles}
\begin{document}


\chapter{The Engineering Project}
\label {ch:engineering}

\sectionnote {AC}
\section{Health and Safety}

This project defines a workflow development platform for creating web applications using Ruby on Rails. Because of the nature of this project there are no health and safety concerns.

\sectionnote {AC}
\section{Engineering Professionalism}

The main goal of this project was to define a platform from which workflow based web applications could be created from. Communication is very important as we must be able to describe in detail the steps that a developer needs to take in order to create a workflow based web application. We provided sufficient documentation for all of the libraries we created and made note of all the existing libraries that we found useful. In addition we have two case studies which provide examples of how each of our libraries work together to easily create a workflow based web application.

Because we have used existing libraries created by other developers licenses were important. All of the externally created libraries come with free licences that allow us to use them for our own work. This means that any web applications created using our platform can be distributed or sold.

\sectionnote {AC}
\section{Project Management}

At the very beginning of the project it was decided that we would test existing workflow frameworks and then build two case studies in an attempt to define common elements and define a standard way to create basic workflow projects. For each case study we used an iterative incremental development process. Iterative incremental development allowed us to gather requirements and test the systems as we built them. This allowed us to identify features that were common between all workflow applications during development and extract the functionalities as we went through the project.

We used the version control system git \cite{git} and github \cite{github} to manage all the code for this project. We made separate repositories for each library we created allowing us to keep our case studies and our libraries separate. This setup allows each library to be individually  maintained separately from the case studies which use them. 

\sectionnote {AC}
\section{Individual Contributions}

This section will go over the contributions of each member describing what they worked on during each stage of the project.

\sectionnote {AC}
\subsection{Project Contributions}

The first part of our project was to test existing workflow web application frameworks. Brendan researched the Stonepath framework and Alexander researched the Ruote framework. We both then created and designed simple issue trackers using the frameworks we researched to see how useful they actually were.

For the first case study Brendan worked on the state machine design and implementation while Alexander worked on account management and project selection. Detailed use cases were created by each member for what they worked on. The testing of the state machines was done by Brendan and the integration testing of the system as a whole was done by Alexander.

In the second case study the roles were switched. Alexander designed and implemented the state machines while Brendan worked on the account management and project selection. Just like in the first case study, detailed use cases were created by each member for what they worked on. Alexander did the state machine testing and integration testing for the second case study.

The final part of our project was the definition of existing libraries and the extraction of useful functionalities to define our platform. Brendan researched and found all of the libraries we used to construct our case studies and also extracted the common functionality of the case studies into their own libraries. Alexander used these libraries using only the documentation in order to ensure they were easy to understand and learn for a new developer.

\sectionnote {AC}
\subsection{Report Contributions}

Each section has the initial of the author of the section listed beside it.

\todo is it enough just to say that each section has name or do we need the list as well.
Brendan:
Chapter 1
Section 3.2.1, 3.2.2
Section 4.1, 4.2
Section 5.1, 5.2, 5.2.1, 5.3, 5.4, 5.5, 5.5.1, 5.5.2, 5.5.3, 5.5.4, 5.5.5, 5.6, 5.6.1, 5.6.2
Section 6.2, 6.4, 6.5, 6.6.1, 6.6.2, 6.6.3

Alexander:
Section 2.1, 2.2, 2.3, 2.4.1, 2.4.2
Section 3.1, 3.2.3
Section 4.3
Section 5.2.2, 5.2.3, 5.2.4, 5.5.6
Section 6.1, 6.3, 6.7

\end{document}
