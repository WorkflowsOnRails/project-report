% -*- compile-command: "cd .. && make" -*-
\documentclass[document.tex]{subfiles}
\begin{document}

\chapter{Platform Test Cases}
\label {ch:platform-test-cases}

\todo {Add an overview of this appendix.}

\sectionnote {BM}
\section {expirable}


\sectionnote {BM}
\section {aasm\_statecharts}
\label {sec:platform-test-cases-aasm_statecharts}

The only parameters are the model class, the filename to save the diagram to, and the format of the diagram.

The characteristics for each parameter are outlined briefly in Tables \ref{tbl:platform-test-cases-aasm_statecharts-aasm-xtics}, \ref{aaa}, and \ref{aaa}. Note that many of the characteristics in Table \ref{tbl:platform-test-cases-aasm_statecharts-aasm-xtics} apply to individual transitions and states, rather than the whole state machine. These characteristics are marked as overlapping as there may be several such characteristics represented in a model. For example, a model may contain both states that include no entry actions, and states that include 2+ entry actions. This extension to standard black-box testing methodology allows us to reduce the number of tests significantly, and test much more reasonable state machines.

\begin{table}[!htbp]
  \centering
  \caption{Characteristics of the model class. Characteristics marked overlapping may have multiple blocks apply to a single test case.}
  \label{tbl:platform-test-cases-aasm_statecharts-aasm-xtics}

  \vspace{3mm}
  \begin{tabular}{l *{3}{c} c}
    \hline
    Characteristic & \multicolumn{3}{c}{Blocks} & Overlapping? \\
    \hline
    AASM included? & No & Yes & & \\
    Number of states & 0 & 1 & 2+ & \\
    Number of transitions & 0 & 1 & 2+ & Yes \\
    Number of transition actions & 0 & 1 & 2+ & Yes \\
    Number of entry actions & 0 & 1 & 2+ & Yes \\
    Number of exit actions & 0 & 1 & 2+ & Yes \\
    Guards used? & No & Yes & & Yes \\
    \hline
  \end{tabular}
\end{table}

\begin{table}[!htbp]
  \centering
  \caption{Characteristics of the filename.}
  \label{tbl:platform-test-cases-aasm_statecharts-filename-xtics}

  \vspace{3mm}
  \begin{tabular}{l *{3}{c} c}
    \hline
    Characteristic & \multicolumn{2}{c}{Blocks} \\
    \hline
    Target file present? & No & Yes \\
    Target directory present? & No & Yes \\
    \hline
  \end{tabular}
\end{table}

\begin{table}[!htbp]
  \centering
  \caption{Characteristics of the file format.}
  \label{tbl:platform-test-cases-aasm_statecharts-format-xtics}

  \vspace{3mm}
  \begin{tabular}{l *{3}{c} c}
    \hline
    Characteristic & \multicolumn{2}{c}{Blocks} \\
    \hline
    File format supported? & No & Yes \\
    \hline
  \end{tabular}
\end{table}

As the underlying library is quite simple, unit tests were selected using the ``all blocks'' criterion. \todo {LIST TESTS, EXPLAIN THAT 100\% COVERAGE WAS ACHIEVED.}

\FloatBarrier

\sectionnote {BM}
\section {aasm\_progressable}

\sectionnote {BM}
\section {aasm\_actionable}


\end {document}